%%%%%%%%%%%%%%%%%%%%%%%%%%%%%%%%%%%%%%%%%
% Beamer Presentation
% LaTeX Template
% Version 1.0 (10/11/12)
%
% This template has been downloaded from:
% http://www.LaTeXTemplates.com
%
% License:
% CC BY-NC-SA 3.0 (http://creativecommons.org/licenses/by-nc-sa/3.0/)
%
%%%%%%%%%%%%%%%%%%%%%%%%%%%%%%%%%%%%%%%%%

%----------------------------------------------------------------------------------------
%	PACKAGES AND THEMES
%----------------------------------------------------------------------------------------

\documentclass[aspectratio=169, 11pt]{beamer}

\mode<presentation> {

% The Beamer class comes with a number of default slide themes
% which change the colors and layouts of slides. Below this is a list
% of all the themes, uncomment each in turn to see what they look like.

%\usetheme{default}
%\usetheme{AnnArbor}
%\usetheme{Antibes}
%\usetheme{Bergen}
%\usetheme{Berkeley}
%\usetheme{Berlin}
%\usetheme{Boadilla}
%\usetheme{CambridgeUS}

\usetheme[progressbar=frametitle]{metropolis}
%\usetheme{Copenhagen}
%\usetheme{Darmstadt}
%\usetheme{Dresden}
%\usetheme{Frankfurt}
%\usetheme{Goettingen}
%\usetheme{Hannover}
%\usetheme{Ilmenau}
%\usetheme{JuanLesPins}
%\usetheme{Luebeck}
%\usetheme{Madrid}
%\usetheme{Malmoe}
%\usetheme{Marburg}
%\usetheme{Montpellier}
%\usetheme{PaloAlto}
%\usetheme{Pittsburgh}
%\usetheme{Rochester}
%\usetheme{Singapore}
%\usetheme{Szeged}
%\usetheme{Warsaw}

% As well as themes, the Beamer class has a number of color themes
% for any slide theme. Uncomment each of these in turn to see how it
% changes the colors of your current slide theme.

%\usecolortheme{albatross}
%\usecolortheme{beaver}
%\usecolortheme{beetle}
%\usecolortheme{crane}
%\usecolortheme{dolphin}
%\usecolortheme{dove}
%\usecolortheme{fly}
%\usecolortheme{lily}
%\usecolortheme{orchid}
%\usecolortheme{rose}
%\usecolortheme{seagull}
%\usecolortheme{seahorse}
%\usecolortheme{whale}
%\usecolortheme{wolverine}

%\setbeamertemplate{footline} % To remove the footer line in all slides uncomment this line
%\setbeamertemplate{footline}[page number] % To replace the footer line in all slides with a simple slide count uncomment this line

%\setbeamertemplate{navigation symbols}{} % To remove the navigation symbols from the bottom of all slides uncomment this line
}

\usepackage{graphicx} % Allows including images
\usepackage{booktabs} % Allows the use of \toprule, \midrule and \bottomrule in tables
\usepackage{amsmath}
\usepackage[english,brazil]{babel}
%\usepackage[latin1]{inputenc}
\usepackage{url,color}
\usepackage{subfigure}
\usepackage{amsthm,amsfonts,amssymb,amscd,amsxtra}
\usepackage{wrapfig}
\usepackage{soul}
\usepackage{xcolor}
\usepackage{array}

%Font
%\usefonttheme{professionalfonts} % using non standard fonts for beamer
%\usefonttheme{serif} % default family is serif
%\setmainfont{Liberation Serif}

%making the section titles appear before each section
\AtBeginSection[]
{
  \begin{frame}
    \frametitle{Table of Contents}
    \setbeamertemplate{section in toc}[sections numbered]
    \vspace{0.3cm}
    \tableofcontents[currentsection]
  \end{frame}
}



%----------------------------------------------------------------------------------------
%	TITLE PAGE
%----------------------------------------------------------------------------------------

\title[]{EC220 -- Introduction to Econometrics} % The short title appears at the bottom of every slide, the full title is only on the title page

\subtitle{Week 2}

\author{Arnaud Dy\`evre} % Your name
\institute[] % Your institution as it will appear on the bottom of every slide, may be shorthand to save space
{
}
\date{October 8\textsuperscript{th} 2020} % Date, can be changed to a custom date

\newtheorem{proposition}{Proposition}
\newtheorem{teo}{Theorem}
\newtheorem{exemplo}{Exemplo}
\newtheorem{corolario}{Corol\'{a}rio}

\newcommand{\R}{\mathbb{R}}
\newcommand{\x}{\textbf{x}}
\newcommand{\y}{\textbf{y}}
\renewcommand{\qedsymbol}{$\blacksquare$}
\newcommand{\dom}{\mathrm{dom}}
\newcommand{\ad}{\mathrm{ad}}
\newcommand{\gerado}{\mathrm{span}}


\begin{document}



\begin{frame}
\titlepage % Print the title page as the first slide
\end{frame}




\begin{frame}{A few words about myself}

\begin{columns}[T]
    \begin{column}{.7\textwidth}
     
\begin{itemize}[<+- | alert@+>]
    \item First name pronounced "\textit{Ar-no}"
    \item 1\textsuperscript{st}-year PhD student
    \item Interests: \textbf{Macroeconomics} and \textbf{Public Economics}
    \item Second time teaching EC220, and I'm excited about it!
    \item Please send me an email if you have any question or doubt: \href{mailto:a.dyevre@lse.ac.uk}{a.dyevre@lse.ac.uk}
    \begin{itemize}
        \item Even if not related to the class
    \end{itemize}
\end{itemize}

    \end{column}
    \begin{column}{.3\textwidth}
% Your image included here
    \includegraphics[width=0.95\textwidth]{photo_alpes.jpeg} \\
    \end{column}
  \end{columns}

\end{frame}

\begin{frame}{A few words about you guys}
    
    \begin{itemize}
        \item Name
        \item Where you're from
        \item Programme of study
    \end{itemize}
\end{frame}

\begin{frame}{Welcome to the course!}

    \begin{itemize}
        \item Challenging at times
        \item \textbf{But} excellent preparation for quant jobs in \fcolorbox{yellow}{yellow}{consulting}, \fcolorbox{yellow}{yellow}{banking}, fields using \fcolorbox{yellow}{yellow}{machine learning} and \fcolorbox{yellow}{yellow}{research}
        \begin{itemize}
            \item Good preparation for learning Python, R and how to think like a data scientist
        \end{itemize}
    \end{itemize}
    
    \vspace{0.5cm}
    \textbf{Course organisation}
    \begin{itemize}[<+- | alert@+>]
        \item Joint teaching for EC220 and EC221 in MT
        \begin{itemize}
            \item Applied part, focus on causal questions
        \end{itemize}
        \item Split up in LT
        \begin{itemize}
            \item Theory, EC221 will use matrix algebra
        \end{itemize}
        \item Textbook: \textit{Mastering Metrics} by Joshua Angrist and Steve Pischke
    \end{itemize}

\end{frame}

\begin{frame}{Welcome to the class!}

    \begin{itemize}[<+- | alert@+>]
        \item 3 classes, all on Thursdays
        \begin{itemize}
            \item \textbf{Group 12:} 10:00-11:00
            \item \textbf{Group 13:} 11:00-12:00
            \item \textbf{Group 11:} 15:00-16:00
        \end{itemize}
        \item Classes start 5 minutes past the hour and end up 5 minutes before
        \item Office hours: Thursdays, 16:00-17:00
        \item Problem sets due the day before the class (Wednesday), at noon \textcolor{red}{(latest)}
        \item \textbf{No cold-calling}, but I expect you to participate in class
        \item Registers: done for every class
        \item Please don't hesitate to interrupt me during the class
        \begin{itemize}
            \item Zoom makes conversations more awkward than in real life and I can't really see when you raise your hand
        \end{itemize}
    \end{itemize}

\end{frame}

\begin{frame}{Welcome to the class!}

    \begin{itemize}[<+- | alert@+>]
        \item \textbf{Letters of recommendation}
        \begin{itemize}
            \item Lecturer of your choice writes it
            \item ...based on information given by class teacher
            \item Really helps if you submit problem sets, show up regularly, and participate in class
        \end{itemize}
        \item \textbf{Submission of problem sets}
        \begin{itemize}
            \item By group from Week 4 onward \textcolor{red}{(see Excel file)}
            \item Submitted through Moodle (1 person for the group)
            \item Informative grade (A, B, C, D) + feedback -- \textbf{Very useful for exam}
        \end{itemize}
        \item \textbf{Forum} on Piazza
        \begin{itemize}
            \item Frequently monitored, don't hesitate to help your fellow students
        \end{itemize}
    \item \textbf{Recording}
    \begin{itemize}
        \item Are you guys happy with this lecture being recorded?
    \end{itemize}
    \end{itemize}

\end{frame}

\begin{frame}{Let's get started with \texttt{Stata}}

    \begin{itemize}[<+- | alert@+>]
        \item Read ``\textbf{EC220\_basic-stata-commands.pdf}"available on Moodle
        \item We will go through ``\textbf{EC220\_intro-to-stata-week2.pdf}" together, available on Moodle
        \item I will use a \texttt{.do} file called ``\texttt{ec220\_week2.do}''
    \end{itemize}

\end{frame}

\begin{frame}{Why \texttt{Stata}?}

    \begin{itemize}[<+- | alert@+>]
        \item Extremely popular among economists
        \item Intuitive
        \item Excellent documentation
        \item A lot of useful, and easily accessible commands for social science
        \item \textbf{Much} better for data cleaning than more general programming languages such as Python
        \item \textbf{But some drawbacks}
\begin{itemize}
    \item Proprietary
    \item Cannot do much more than statistical analysis
    \item R and Python community develops packages more rapidly (but with a lot of bugs sometimes)
\end{itemize}
    \end{itemize}
    
\vspace{0.5cm}


\end{frame}

\begin{frame}{Computer exercise}

\begin{center}
    Please open Stata on your computer and follow along
\end{center}
    
\end{frame}

\begin{frame}{Commands you will use frequently}

\begin{itemize}
    \item \texttt{clear all}
\item \texttt{set more off}
\item \texttt{use filename [, clear]}
\item \texttt{cd}
\item \texttt{describe}
\item \texttt{list [varlist] [if] [in]}
\item \texttt{summarize [varlist] [if] [in][, detail]}
\item \texttt{tabulate varname [if] [in] [, row col]}
\item \texttt{generate newvar = exp [if] [in]}
\item \texttt{replace oldvar = exp [if] [in]}
\item \texttt{save filename [, replace]}

\end{itemize}

\end{frame}

\begin{frame}{Next week}

\begin{itemize}
    \item Another \texttt{Stata} session
    \begin{itemize}
        \item Using a dataset frequently used in Machine Learning competitions (such as \href{https://www.kaggle.com/c/titanic}{Kaggle})
    \end{itemize}
    \item Done individually
    \item You will use what we have learned today
\end{itemize}
    
\end{frame}

\begin{frame}{Reminder}

\textbf{Office hours:} Thursdays, 16:00-17:00\\

Zoom link on Moodle

\vspace{1cm}

\begin{center}
    \textcolor{red}{Have a great week!}
\end{center}
    
\end{frame}

\end{document}